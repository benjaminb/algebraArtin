\documentclass[main.tex]{subfiles}
% \newcommand{\uequals}{\mathrel{\rotatebox[origin=c]{45}{$=$}}}
% \newcommand{\dequals}{\mathrel{\rotatebox[origin=c]{135}{$=$}}}
% \newcommand{\vequals}{\mathrel{\rotatebox[origin=c]{90}{$=$}}}
\begin{document}
\section*{Chapter 2}
% \documentclass{article}

% \usepackage{graphicx}
% \usepackage{amsmath}

% \begin{document}

% \title{Algebra Chapter 2}
% \author{Your Name}
% \date{\today}

% \maketitle

\section{Section 1}

\begin{enumerate}
    \item Let $S$ be a set. Prove that the law of composition defined by $ab = a$ for all $a, b$ in $S$ is associative.

    \textbf{Solution: use algebra to demonstrate order of groupings doesn't matter}
    We'll show that $a \circ (b \circ c) = (a \circ b) \circ c$. For any $a, b, c \in S$, the composition $a \circ (b \circ c) = a\circ b = a$. Also $(a \circ b) \circ c = a \circ c = a$. Therefore this composition is associative. 

    Takeaway: projection is associative.

    \item. Prove the properties of inverses that are listed near the end of the section. 
    
    The properties are:
    
    \begin{itemize}
        \item If an element $a$ has both a left inverse $\ell$ and a right inverse $r$, then the left inverse and the right inverse are equal.
        
        \textbf{Solution: use algebra to show equality}. We'll use algebra to show that $\ell = r$. Given $a\ell = 1$, multiply both sides by $r$. This gives $\ell ar = 1r = r$. Using associativity we have $\ell(ar)= r$. But since $ar = 1$, as $r$ is a right inverse, this simplifies to $\ell = r$. 
        \item If $a$ is invertible, its inverse is unique.
        \textbf{Solution: assume there are multiple inverses and show they are equal.}

        Suppose $a$ is invertible. This means there exists at least one inverse $a^{-1}$. Let $b, c$ be other, possibly different inverses for $a$. Then
        $$ca = ba \implies caa^{-1} = baa^{-1} \implies c = b$$

        showing that $b = c$.

        \item Inverses multiply in the opposite order: $(ab)^{-1} = b^{-1}a^{-1}$.
        
        \textbf{Solution: use algebra to demonstrate the property holds.}

        This requires that $a$ and $b$ individually have inverses. Using associativity:
        $$ab(b^{-1}a^{-1}) = a(bb^{-1})a^{-1} = a1a^{-1} = aa^{-1} = 1$$

        \item An element $a$ may have a left inverse or a right inverse, though it is not invertible.
        
        \textbf{Solution: show an example}
        
        \item Let $\mathbb{N}$ denote the set $\{1, 2, 3, \ldots\}$, and let $s: \mathbb{N} \to \mathbb{N}$ be the \textit{shift} map function defined by $s(n) = n + 1$. Prove that $s$ has no right inverse, but that it has infinitely many left inverses. 
        \textbf{Solution: explore the function's properties and form a proof.}

        Suppose $s$ has a right inverse $t$. Then $t$ must be a function from $\mathbb{N}$ to $\mathbb{N}$ such that $s(t(n)) = n$. We can also write this as $t(n) + 1 = n$. This clearly works if $t(n) = n - 1$. However if $n = 1$ then $n - 1$ is not in the target set $\mathbb{N}$. In this sense $s$ cannot have a right inverse. 

        If $s$ has a left inverse then $t(s(n)) = n$ for all $n$. I.e. $t(n + 1) = n$. We can define $t(n) = n - 1$ and since $t$ can never receive a number lower than 2 as input, we don't have the problem before of $t(n)$ mapping out of $\mathbb{N}$. However, $t$ must still be \textit{defined} on the entire set $\mathbb{N}$, so we can map 1 to any natural number we like. For example we could have $t_1$ which maps 1 to 1, $t_{29}$ which maps 1 to 29, etc. In this sense each $t_k$ is a left inverse of $s$ and there are infinitely many different ones, one for each natural number.
    \end{itemize}
\end{enumerate}

\section{Section 2}

\begin{enumerate}[label=2.\arabic*]
    \item First
    \item Second
    \item third
    \item fourth
    \item In the definition of a subgroup, the identity element in $H$ is required to be the identity of $G$. One might require only that $H$ have an identity element, not that it need be the same as the identity of $G$. Show that if $H$ has an identity at all, then it is the identity of $G$. Show the analagous statement is true for inverses.
    
    \textbf{Solution: isolate the essential difference between the objects then show they must be equal.}
    Suppose $H \leq G$ and $e$ is the identity of $H$. Then for any $h \in H, eh = he = h$. Now take some $g \in G$ that is not in $H$ (if there is no such $g$ then $H = G$ and $e$ is the identity in $G$, as the group identity is unique). Let $eg = g'$. Perhaps $g \neq g'$ and $e$ is not the identity in $G$. However, using associativity:

    $$
    \begin{array}{cllc}
    & & (eh)g \ = &hg \\
    & \uequals & &\\
    ehg & & & \vequals\\
    & \dequals & & \\
    & & (eh)g = (he)g = h(eg) = &hg' \\
    \end{array}
    $$

    shows $hg = hg'$. We apply the cancellation law to get $g = g'$. That means $eg = g$ for any $g \in G$ not in $H$ and $e$ is already the identity for $H$. Therefore $e$ behaves as the identity in $G$ as well, and since group identities are unique, $e$ must be the identity of $G$.

    For inverses, suppose $h \in H$ has an inverse $h^{-1}$ in $H$ but a possibly different inverse $j \in G$. Then in $G$, $hh^{-1} = hj$ and by the cancellation law $h^{-1} = j$. 

\end{enumerate}



\end{document}